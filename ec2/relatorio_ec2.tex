\documentclass[12pt, a4paper]{article}
\usepackage[bottom=2cm,top=3cm,left=3cm,right=2cm]{geometry}
\usepackage[brazilian]{babel} % Traduz alguns termos para o português
\usepackage[utf8]{inputenc} % Reconhece acentuação
\usepackage{color,graphicx}
\usepackage{enumerate}
\usepackage{mathtools}
\usepackage{listings}
\usepackage{longtable}
\usepackage[section]{placeins}
\usepackage[hidelinks=true]{hyperref}
\hypersetup{
	colorlinks=false,     
	urlcolor=blue
}

\definecolor{olive}{RGB}{175,128,0}
\definecolor{Aquamarine}{RGB}{0,175,175}

\usepackage{setspace}
\onehalfspacing	
\setlength{\parindent}{30pt}

\usepackage{indentfirst}

\title{
	\begin{large}
		Estudo de Caso 02: Comparação de Duas Amostras
	\end{large}	}
\author{Gustavo Vieira Costa - 2010022003\\Rafael Castro - 2013030210\\Thaís Matos Acácio - 2013030287}
\date{08/04/2016}

\begin{document}
	\maketitle
	
	\vspace*{-7.5cm}
	{\bf
		\begin{center}
			{\large
				\hspace*{0cm}Universidade Federal de Minas Gerais} \\
			\hspace*{0cm}Engenharia de Sistemas  \\
		\end{center}
	}
	\vspace*{5cm}
	
\section{Introdução}
O BMI (\textit{body mass index}, ou índice de massa corporal) é um indicador frequentemente usado em avaliações clínicas de questões relacionadas ao peso de um indivíduo. Este índice é calculado como a razão entre o peso e o quadrado da estatura.
\par O objetivo desse experimento é comparar o BMI médio de duas populações de estudantes: alunos de graduação em Engenharia de Sistemas e alunos de pós-graduação em Engenharia Elétrica, com interesse de relacionar o efeito do curso na forma física dos alunos.

\section{Coleta de dados}
Resultados a partir de uma amostra aleatória obedecem as leis de probabilidade, as quais governam o comportamento aleatório e permitem inferência confiável sobre a população.
\par A Tabela \ref{table:amostra} contém a amostra de dados coletados, informados pelos alunos de cada turma, juntamente com o valor do índice BMI calculado utilizando a seguinte fórmula:
\begin{equation}
bmi = \frac{m}{h^{2}}
\end{equation}
\newline onde \textit{m} é o peso dado em kg e \textit{h} a altura dada em metros.
\pagebreak
\begin{longtable}{|c|c|c|c|c|}
%\begin{tabular}{|c|c|c|c|c|}
\hline
\rule[-1.0ex]{0pt}{4.0ex}
\textbf{Curso}&\textbf{ID}&\textbf{Altura(m)}&\textbf{Peso(kg)}&\textbf{BMI}\\ \hline
\endhead
\rule[-1.0ex]{0pt}{4.0ex}
PPGEE&PG-ST1&1.83&77& \\ \hline
\rule[-1.0ex]{0pt}{4.0ex}
PPGEE&PG-ST2&1.67&56& \\ \hline
\rule[-1.0ex]{0pt}{4.0ex}
PPGEE&PG-ST3&1.88&86& \\ \hline
\rule[-1.0ex]{0pt}{4.0ex}
PPGEE&PG-ST4&1.77&78& \\ \hline
\rule[-1.0ex]{0pt}{4.0ex}
PPGEE&PG-ST5&1.74&78& \\ \hline
\rule[-1.0ex]{0pt}{4.0ex}
PPGEE&PG-ST6&1.98&113& \\ \hline
\rule[-1.0ex]{0pt}{4.0ex}
PPGEE&PG-ST7&1.70&77& \\ \hline
\rule[-1.0ex]{0pt}{4.0ex}
PPGEE&PG-ST8&1.81&78& \\ \hline
\rule[-1.0ex]{0pt}{4.0ex}
PPGEE&PG-ST9&1.55&54& \\ \hline
\rule[-1.0ex]{0pt}{4.0ex}
PPGEE&PG-ST10&1.82&96& \\ \hline
\rule[-1.0ex]{0pt}{4.0ex}
PPGEE&PG-ST11&1.81&73& \\ \hline
\rule[-1.0ex]{0pt}{4.0ex}
PPGEE&PG-ST12&1.65&61& \\ \hline
\rule[-1.0ex]{0pt}{4.0ex}
PPGEE&PG-ST13&1.65&60& \\ \hline
\rule[-1.0ex]{0pt}{4.0ex}
PPGEE&PG-ST14&1.73&76& \\ \hline
\rule[-1.0ex]{0pt}{4.0ex}
PPGEE&PG-ST15&1.75&85& \\ \hline
\rule[-1.0ex]{0pt}{4.0ex}
PPGEE&PG-ST16&1.81&74& \\ \hline
\rule[-1.0ex]{0pt}{4.0ex}
PPGEE&PG-ST17&1.82&67& \\ \hline
\rule[-1.0ex]{0pt}{4.0ex}
PPGEE&PG-ST18&1.70&64& \\ \hline
\rule[-1.0ex]{0pt}{4.0ex}
PPGEE&PG-ST19&1.65&64& \\ \hline
\rule[-1.0ex]{0pt}{4.0ex}
PPGEE&PG-ST20&1.75&88& \\ \hline
\rule[-1.0ex]{0pt}{4.0ex}
PPGEE&PG-ST21&1.85&96& \\ \hline
\rule[-1.0ex]{0pt}{4.0ex}
PPGEE&PG-ST22&1.83&85& \\ \hline
\rule[-1.0ex]{0pt}{4.0ex}
PPGEE&PG-ST23&1.78&58& \\ \hline
\rule[-1.0ex]{0pt}{4.0ex}
PPGEE&PG-ST24&1.70&72& \\ \hline
\rule[-1.0ex]{0pt}{4.0ex}
PPGEE&PG-ST25&1.70&65& \\ \hline
\rule[-1.0ex]{0pt}{4.0ex}
PPGEE&PG-ST26&1.72&98& \\ \hline
\rule[-1.0ex]{0pt}{4.0ex}
PPGEE&PG-ST27&1.67&53& \\ \hline
\rule[-1.0ex]{0pt}{4.0ex}
PPGEE&PG-ST28&1.79&78& \\ \hline
\rule[-1.0ex]{0pt}{4.0ex}
EngSis&ES-ST1&1.56&48& \\ \hline
\rule[-1.0ex]{0pt}{4.0ex}
EngSis&ES-ST2&1.67&61.5& \\ \hline
\rule[-1.0ex]{0pt}{4.0ex}
EngSis&ES-ST3&1.68&60& \\ \hline
\rule[-1.0ex]{0pt}{4.0ex}
EngSis&ES-ST4&1.65&63& \\ \hline
\rule[-1.0ex]{0pt}{4.0ex}
EngSis&ES-ST5&1.69&57& \\ \hline
\rule[-1.0ex]{0pt}{4.0ex}
EngSis&ES-ST6&1.83&80& \\ \hline
\rule[-1.0ex]{0pt}{4.0ex}
EngSis&ES-ST7&1.71&76& \\ \hline
\rule[-1.0ex]{0pt}{4.0ex}
EngSis&ES-ST8&1.71&70& \\ \hline
\rule[-1.0ex]{0pt}{4.0ex}
EngSis&ES-ST9&1.65&70& \\ \hline
\rule[-1.0ex]{0pt}{4.0ex}
EngSis&ES-ST10&1.83&66& \\ \hline
\rule[-1.0ex]{0pt}{4.0ex}
EngSis&ES-ST11&1.64&52& \\ \hline
\rule[-1.0ex]{0pt}{4.0ex}
EngSis&ES-ST12&1.78&68& \\ \hline
\rule[-1.0ex]{0pt}{4.0ex}
EngSis&ES-ST13&1.76&82.5& \\ \hline
%\end{tabular}
\caption{Tabela de Amostras}
\label{table:amostra}
\end{longtable}

\section{Estratégia de Inferência}
\label{sec:estrategia-inferencia}

\section{Projeto experimental}
\label{sec:projeto-experimental}

\section{Análise dos Resultados}

\section{Conclusão}

\begin{thebibliography}{7}
\bibitem{1}{\url{https://github.com/fcampelo/Design-and-Analysis-of-Experiments}}
\bibitem{2}{Estatística Aplicada e Probabilidade para Engenheiros (4ª edição) - Montgomery}
\bibitem{3}{A Estatística Básica e Sua Prática (6ª edição) - David S. Moore, William I. Nortz, Michael A. Fligner}
\bibitem{4}{\url{https://www.youtube.com/watch?v=SacXljL9dKQ&nohtml5=False}}
\bibitem{5}{\url{https://www.youtube.com/watch?v=TJbnkmiZiRU&nohtml5=False}}
\bibitem{6}{\url{https://stat.ethz.ch/R-manual/R-devel/library/stats/html/var.test.html}}
\bibitem{7}{\url{http://ww2.coastal.edu/kingw/statistics/R-tutorials/independent-t.html}}
\end{thebibliography}		
		
\end{document}