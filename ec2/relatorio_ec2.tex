\documentclass[12pt, a4paper]{article}
\usepackage[bottom=2cm,top=3cm,left=3cm,right=2cm]{geometry}
\usepackage[brazilian]{babel} % Traduz alguns termos para o português
\usepackage[utf8]{inputenc} % Reconhece acentuação
\usepackage{color,graphicx}
\usepackage{enumerate}
\usepackage{mathtools}
\usepackage{listings}
\usepackage{longtable}
\usepackage[section]{placeins}
\usepackage[hidelinks=true]{hyperref}
\hypersetup{
	colorlinks=false,     
	urlcolor=blue
}

\definecolor{olive}{RGB}{175,128,0}
\definecolor{Aquamarine}{RGB}{0,175,175}

\usepackage{setspace}
\onehalfspacing	
\setlength{\parindent}{30pt}

\usepackage{indentfirst}

\title{
	\begin{large}
		Estudo de Caso 02: Comparação de Duas Amostras
	\end{large}	}
\author{Gustavo Vieira Costa - 2010022003\\Rafael Castro - 2013030210\\Thaís Matos Acácio - 2013030287}
\date{08/04/2016}

\begin{document}
	\maketitle
	
	\vspace*{-7.5cm}
	{\bf
		\begin{center}
			{\large
				\hspace*{0cm}Universidade Federal de Minas Gerais} \\
			\hspace*{0cm}Engenharia de Sistemas  \\
		\end{center}
	}
	\vspace*{5cm}
	
\section{Introdução}
O BMI (\textit{body mass index}, ou índice de massa corporal) é um indicador frequentemente usado em avaliações clínicas de questões relacionadas ao peso de um indivíduo. Este índice é calculado como a razão entre o peso e o quadrado da estatura.
\par O objetivo desse experimento é comparar o BMI médio de duas populações de estudantes: alunos de graduação em Engenharia de Sistemas e alunos de pós-graduação em Engenharia Elétrica, com interesse de relacionar o efeito do curso na forma física dos alunos.

\section{Coleta de dados}
Resultados a partir de uma amostra aleatória obedecem as leis de probabilidade, as quais governam o comportamento aleatório e permitem inferência confiável sobre a população.
\par A Tabela \ref{table:amostra} contém a amostra de dados coletados, informados pelos alunos de cada turma, juntamente com o valor do índice BMI calculado utilizando a seguinte fórmula:
\begin{equation}
bmi = \frac{m}{h^{2}}
\end{equation}
\newline onde \textit{m} é o peso dado em kg e \textit{h} a altura dada em metros. A distribuição do BMI, de acordo com o curso, foi exibida no gráfico de blocos \ref{fig:blocos} para possibilitar uma melhor visualização da diferença entre as amostras.
\pagebreak
\begin{longtable}{|c|c|c|c|c|}
\hline
\rule[-1.0ex]{0pt}{4.0ex}
\textbf{Curso}&\textbf{ID}&\textbf{Altura(m)}&\textbf{Peso(kg)}&\textbf{BMI}\\ \hline
\endhead
\rule[-1.0ex]{0pt}{4.0ex}
PPGEE&PG-ST1&1.83&77&22.99 \\ \hline
\rule[-1.0ex]{0pt}{4.0ex}
PPGEE&PG-ST2&1.67&56&20.08 \\ \hline
\rule[-1.0ex]{0pt}{4.0ex}
PPGEE&PG-ST3&1.88&86&24.33 \\ \hline
\rule[-1.0ex]{0pt}{4.0ex}
PPGEE&PG-ST4&1.77&78&24.90 \\ \hline
\rule[-1.0ex]{0pt}{4.0ex}
PPGEE&PG-ST5&1.74&78&25.76 \\ \hline
\rule[-1.0ex]{0pt}{4.0ex}
PPGEE&PG-ST6&1.98&113&28.82 \\ \hline
\rule[-1.0ex]{0pt}{4.0ex}
PPGEE&PG-ST7&1.70&77&26.64 \\ \hline
\rule[-1.0ex]{0pt}{4.0ex}
PPGEE&PG-ST8&1.81&78&23.81 \\ \hline
\rule[-1.0ex]{0pt}{4.0ex}
PPGEE&PG-ST9&1.55&54&22.48 \\ \hline
\rule[-1.0ex]{0pt}{4.0ex}
PPGEE&PG-ST10&1.82&96&28.98 \\ \hline
\rule[-1.0ex]{0pt}{4.0ex}
PPGEE&PG-ST11&1.81&73&22.28 \\ \hline
\rule[-1.0ex]{0pt}{4.0ex}
PPGEE&PG-ST12&1.65&61&22.41 \\ \hline
\rule[-1.0ex]{0pt}{4.0ex}
PPGEE&PG-ST13&1.65&60&22.04 \\ \hline
\rule[-1.0ex]{0pt}{4.0ex}
PPGEE&PG-ST14&1.73&76&25.39 \\ \hline
\rule[-1.0ex]{0pt}{4.0ex}
PPGEE&PG-ST15&1.75&85&27.76 \\ \hline
\rule[-1.0ex]{0pt}{4.0ex}
PPGEE&PG-ST16&1.81&74&22.59 \\ \hline
\rule[-1.0ex]{0pt}{4.0ex}
PPGEE&PG-ST17&1.82&67&20.23 \\ \hline
\rule[-1.0ex]{0pt}{4.0ex}
PPGEE&PG-ST18&1.70&64&22.15 \\ \hline
\rule[-1.0ex]{0pt}{4.0ex}
PPGEE&PG-ST19&1.65&64&23.51 \\ \hline
\rule[-1.0ex]{0pt}{4.0ex}
PPGEE&PG-ST20&1.75&88&28.73 \\ \hline
\rule[-1.0ex]{0pt}{4.0ex}
PPGEE&PG-ST21&1.85&96&28.05 \\ \hline
\rule[-1.0ex]{0pt}{4.0ex}
PPGEE&PG-ST22&1.83&85&25.38 \\ \hline
\rule[-1.0ex]{0pt}{4.0ex}
PPGEE&PG-ST23&1.78&58&18.31 \\ \hline
\rule[-1.0ex]{0pt}{4.0ex}
PPGEE&PG-ST24&1.70&72&24.91 \\ \hline
\rule[-1.0ex]{0pt}{4.0ex}
PPGEE&PG-ST25&1.70&65&22.49 \\ \hline
\rule[-1.0ex]{0pt}{4.0ex}
PPGEE&PG-ST26&1.72&98&33.13 \\ \hline
\rule[-1.0ex]{0pt}{4.0ex}
PPGEE&PG-ST27&1.67&53&19.00 \\ \hline
\rule[-1.0ex]{0pt}{4.0ex}
PPGEE&PG-ST28&1.79&78&24.34 \\ \hline
\rule[-1.0ex]{0pt}{4.0ex}
EngSis&ES-ST1&1.56&48&19.72 \\ \hline
\rule[-1.0ex]{0pt}{4.0ex}
EngSis&ES-ST2&1.67&61.5&22.05 \\ \hline
\rule[-1.0ex]{0pt}{4.0ex}
EngSis&ES-ST3&1.68&60&21.26 \\ \hline
\rule[-1.0ex]{0pt}{4.0ex}
EngSis&ES-ST4&1.65&63&23.14 \\ \hline
\rule[-1.0ex]{0pt}{4.0ex}
EngSis&ES-ST5&1.69&57&19.96 \\ \hline
\rule[-1.0ex]{0pt}{4.0ex}
EngSis&ES-ST6&1.83&80&23.89 \\ \hline
\rule[-1.0ex]{0pt}{4.0ex}
EngSis&ES-ST7&1.71&76&25.99 \\ \hline
\rule[-1.0ex]{0pt}{4.0ex}
EngSis&ES-ST8&1.71&70&23.94 \\ \hline
\rule[-1.0ex]{0pt}{4.0ex}
EngSis&ES-ST9&1.65&70&25.71 \\ \hline
\rule[-1.0ex]{0pt}{4.0ex}
EngSis&ES-ST10&1.83&66&19.71 \\ \hline
\rule[-1.0ex]{0pt}{4.0ex}
EngSis&ES-ST11&1.64&52&19.33 \\ \hline
\rule[-1.0ex]{0pt}{4.0ex}
EngSis&ES-ST12&1.78&68&21.46 \\ \hline
\rule[-1.0ex]{0pt}{4.0ex}
EngSis&ES-ST13&1.76&82.5&26.63 \\ \hline
\caption{Tabela de Amostras}
\label{table:amostra}
\end{longtable}
\begin{figure}[h]
\centering
\includegraphics[scale=0.6]{img/boxplot.png}
\caption{Distribuição das amostras}
\label{fig:blocos}
\end{figure}
\par De acordo com o \textit{Teorema do Limite Central}, se a amostra tiver tamanho $n$ suficiente, a distribuição amostral de $\bar{x}$ é aproximadamente Normal. Nesse caso, iremos assumir que $n_{1}$ e $n_{2}$ são suficientes, conforme o gráfico de normalidade presente na figura \ref{fig:normal}.
\begin{figure}[h]
\centering
\includegraphics[scale=0.8]{img/plot-normalidade.png}
\caption{Normalidade das amostras}
\label{fig:normal}
\end{figure}

\section{Estratégia de Inferência}
\label{sec:estrategia-inferencia}
O processo de inferência estatística consiste em tirar conclusões sobre uma população com base em informações extraídas de amostras da mesma. No presente estudo de caso, os parâmetros sobre os quais temos interesse são as médias $\mu_{1}$ e $\mu_{2}$ do índice BMI das populações: alunos de graduação em Engenharia de Sistemas e alunos de pós graduação em Engenharia Elétrica.
\newline
\newline
\textbf{Premissas do Teste}
\begin{itemize}
\item O \textbf{nível de significância}($\alpha$), representa a probabilidade de erro tipo I, ou seja, a probabilidade de rejeitarmos a hipótese nula quando ela é efetivamente verdadeira. Pensando em uma taxa de erro aceitável para o domínio do problema, fixamos $\alpha = 0.05$.
\item \textbf{Nível de confiança} ($1 - \alpha$) tem como objetivo conhecer o quanto o teste de hipóteses controla um erro do tipo I, ou qual a probabilidade de aceitar a hipótese nula se realmente for verdadeira.
\item O menor \textbf{tamanho de efeito} de importância prática($\delta^*$) como 1.5, tomando como referência os valores da tabela de classificação do BMI (30\% do tamanho máximo das categorias).
\end{itemize}
\par A probabilidade, calculada supondo-se $H_{0}$ verdadeira, de que a estatística de teste assuma um valor tão ou mais extremo do que o valor realmente observado é chamada de \textbf{valor P}.

\textbf{Hipóteses de Teste}
\par O teste estatístico é planejado para avaliar a força da evidência \textit{contra} a hipótese nula $H_{0}$. Usualmente, a hipótese nula é uma afirmativa de "nenhum efeito". A afirmativa sobre a população \textit{a favor} da qual estamos tentando achar evidência é a hipótese alternativa $H_{1}$. Logo, as hipóteses são:
\begin{equation}
\left \{
\begin{array}{cc}
H_{0}: & \mu_{1} = \mu_{2} \\
H_{1}: & \mu_{1} \neq \mu_{2} \\
\end{array}
\right.
\end{equation}
\newline sendo $\mu_{1}$ a média da população do PPGEE, e $\mu_{2}$ a média dos alunos de graduação em Engenharia de Sistemas.
\par Para escolha do teste estatístico que será utilizado no projeto inicialmente é preciso estudar se existe igualdade entre as variâncias das duas populações, para isso podemos utilizar o teste F assumindo como hipótese nula a igualdade de duas variâncias, portanto:
\begin{equation}
\left \{
\begin{array}{cc}
H_{0}: \sigma_{1}^{2} = \sigma_{2}^{2} \\
H_{1}: \sigma_{1}^{2} \neq \sigma_{2}^{2} \\
\end{array}
\right.
\end{equation}
Os resultados do teste F definem se utilizaremos o teste t para duas amostras, caso onde as variâncias podem ser consideradas iguais(hipótese nula não rejeitada), ou o teste Welch, caso contrário.
\newline
\newline
\textit{F test to compare two variances} \newline
data:  bmiEngSis and bmiPpgee \newline
F = 0.5822, num df = 12, denom df = 27, p-value = 0.3256 \newline
alternative hypothesis: true ratio of variances is not equal to 1 \newline
95 percent confidence interval: \newline
 0.2358306 1.7396886 \newline
sample estimates: \newline
ratio of variances \newline
         0.5822072 \newline

\par Como o valor observado de F não pertence à região crítica encontrada pelo \textit{var.test} e o p-value é maior que o nível de significância $\alpha$, então não existem evidências suficientes para rejeitar H0, portanto iremos considerar a igualdade das variâncias e utilizar o teste t para duas amostras.

\section{Projeto experimental e Análise dos Resultados}
\label{sec:projeto-experimental}
Aplicando o teste t bilateral para duas amostras, obtemos o seguinte resultado:
\newline
\newline
\textit{Two Sample t-test} \newline
data:  bmiEngSis and bmiPpgee \newline
t = -1.7259, df = 39, p-value = 0.09228 \newline
alternative hypothesis: true difference in means is not equal to 0 \newline
95 percent confidence interval: \newline
 -3.9443195  0.3122583 \newline
sample estimates: \newline
mean of x mean of y \newline
 22.52300  24.33903 \newline
\par O teste de potência, considerando variâncias iguais e o tamanho de efeito $\alpha$=1.5, apresentou o seguinte comportamento:
\newline
\newline
\textit{t test power calculation } \newline
             n1 = 13 \newline
             n2 = 28 \newline
              d = 1.5 \newline
      sig.level = 0.05 \newline
          power = 0.9917339 \newline
    alternative = two.sided \newline

\par Com base nesses resultados, concluimos que a probabilidade de ocorrer erro tipo II é de 0.01 aproximadamente. O poder do teste apresentou um valor elevado devido à estimativa do tamanho de efeito inicial. Se tivéssemos estimado $\delta^*$ com um valor menor, obteríamos um poder também menor.

\section{Conclusão}
A partir da análise do experimento, chegamos à conclusão que as médias dos BMI's dos alunos dos cursos de graduação em  Engenharia de Sistemas e Pós-graduação em Engenharia Elétrica podem ser consideradas iguais, visto que não temos evidências suficientes para rejeitar a hipótese nula proposta.
\par Mesmo que $n_{1}$ fosse reduzido para 12, ou $n_{2}$ para 25, o poder do teste se manteria inalterado, conforme resultados abaixo:
\newline
\newline
\textit{t test power calculation} \newline
             n1 = 12.41101 \newline
             n2 = 28 \newline
              d = 1.5 \newline
      sig.level = 0.05 \newline
          power = 0.99 \newline
    alternative = two.sided \newline
\newline
\textit{t test power calculation} \newline
             n1 = 13 \newline
             n2 = 25.60339 \newline
              d = 1.5 \newline
      sig.level = 0.05 \newline
          power = 0.99 \newline
    alternative = two.sided \newline
\newline
\par Algumas possíveis formas de melhorar a qualidade do experimento:
\begin{itemize}
\item Melhorar o método de coleta de dados, pois a qualidade dos dados tem impacto direto na qualidade do experimento.
\end{itemize}

\begin{thebibliography}{6}
\bibitem{1}{\url{https://github.com/fcampelo/Design-and-Analysis-of-Experiments}}
\bibitem{2}{Estatística Aplicada e Probabilidade para Engenheiros (4ª edição) - Montgomery}
\bibitem{3}{A Estatística Básica e Sua Prática (6ª edição) - David S. Moore, William I. Nortz, Michael A. Fligner}
\bibitem{4}{\url{https://stat.ethz.ch/R-manual/R-devel/library/stats/html/var.test.html}}
\bibitem{5}{\url{http://ww2.coastal.edu/kingw/statistics/R-tutorials/independent-t.html}}
\bibitem{6}{\url{http://www.portalaction.com.br/inferencia/56-teste-para-comparacao-de-duas-variancias-teste-f}}
\end{thebibliography}		
		
\end{document}